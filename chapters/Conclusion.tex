\chapter*{Major findings and Conclusion}
\phantomsection
\addcontentsline{toc}{chapter}{Major findings and Conclusion}

The Russian cyber operation during the wartime in Ukraine revealed several key insights and lessons. Although the integration of cyber capabilities in military operation is in early stages and far from being structured with clear doctrine and procedures, there have been several successful attempts on this side from the Russian forces, such as attacks on Viasat and the power grid, to prepare or in conjunction with kinetic operations. However, this case, the first interstate full scale war in the information age. By analysing the Russian cyber operations during this conflict, the research has provided a comprehensive understanding of the challenges and opportunities presented by cyber capabilities in modern warfare. 

The case study has stressed the necessity to implement a coherent and resilient strategy with respect to critical infrastructure to prevent future offensive cyber capabilities from causing destruction over the disruption. A more proactive role of the victim has been analysed, both with the use of preventive and offensive cyber capabilities and the building of a buoyant defence to deter future attacks. However, the challenging investment in skills and infrastructure makes it difficult to achieve not only for Ukraine but also for other developed countries. 

The research has shown that the European Union, even without a military power, constitutes a \textit{normative cyber power}, which helps countries coordinate and comply with their national regulations. Furthermore, through its funds and projects (such as the PESCO), it has driven countries to innovate in the cyber domain and to train their cyber capabilities in both military and civil missions. In addition, its diplomatic position gives a univocal voice for its Member States, which leverage the cyber diplomacy as an answer, method, and instrument on the cyber domain. With a projection on being more relevant even on the defence domain, The EU's role as a catalyst in civilian aspects, critical infrastructure certification, and cross-country risk management is crucial more than ever. 

The comparative analysis of France, Estonia, Germany, and the Netherlands reveals a multifaceted approach to cybersecurity, acknowledging the lessons learned from the war in Ukraine. 

In the realm of political initiatives and deterrence, France places emphasis on political alliances and collaboration with the private sector, recognizing the critical role of collective defence efforts. Estonia, on the other hand, adopts a deterrence-through-denial strategy and actively participates in international cooperation, hosting key cyber defence institutions. Germany focuses on boosting military capabilities and engages in collaborative governance for strategic military positioning. The Netherlands stands out with its robust cybersecurity infrastructure, diplomatic responses to cyberattacks, and active participation in international alliances.

Economic measures and protection of critical infrastructure emerge as crucial pillars of cybersecurity resilience. Both France and Germany allocate resources to address the skills gap and improve infrastructure, demonstrating a commitment to overcoming human resource challenges. Estonia invests in skills and infrastructure, leveraging its Cyber Command subdivisions for skill development and coordination. The Netherlands establishes a national network for information sharing and places a strong emphasis on secure design principles in safeguarding critical infrastructure.

In terms of military integration and offensive capabilities, France adopts a proactive stance, integrating cyber-kinetic capabilities and employing preventive offensive measures. Estonia actively participates in cyber exercises that explore kinetic-cyber synergies and engages in offensive cyber exercises. Germany, meanwhile, delegates preventive intelligence gathering to counter military threats, with no explicit mention of offensive postures. The Netherlands takes a proactive approach, promoting joint cyber mission teams and actively developing offensive capabilities.

Challenges and future directions are also evident across these nations. A common challenge is the skills gap, prompting various initiatives and collaborations to address this critical issue. The cooperation between public and private sectors remains a challenge, with tailored projects in Estonia and the Netherlands specifically targeting SMEs. The war in Ukraine has influenced defence strategies, with increased funding and resource allocation observed in Germany and the Netherlands.

\section*{Is the EU a Cyber Power?}

Does the European Union be a cyber power and project its vision not only in the region but also worldwide? The academic community has often tried to systemically understand the relationship between cyber and power, but without success. According to \textcite{dunncavelty_2018_europes}, there is no systemic empirical analysis of the topic, which is the fragmented and clear dominance of the US vision. For this reason, a new analytical framework is needed to systemically study the behaviour of international actors in cyberspace. 


While the EU has increased its international projection in different areas, such as international trade and human rights, security, and foreign policy aspects have developed less intensely. In fact, owing to the nature of the EU, it is difficult to observe how it could be recognised as cyber power. \textit{Does the EU use cyberspace to create advantages and influence events in other operational environments and across instruments of power}? \autocite[12]{kuehl_2009_from}. Although we have emphasised that the securitisation of cyberspace is mainly the responsibility of sovereign states, the EU's approach to cybersecurity identifies new paths that go beyond the traditional notion of state security. The concept of power will be meaningless without the specification of the context \autocite[454]{guzzini_1993_structural}. 

Measuring cyber power is a challenging undertaking. There is a lack of shared methods and frameworks between practitioners and scholars to assess the extent of cyber power. The distinctive features of cyberspace and its elements are one of the main causes of this situation. Although assessing military power can involve quantifying physical capabilities such as submarines, tanks, and others, the same approach cannot be applied to cyberspace due to its layered structure, where only one layer is directly connected to the physical world. Most cyber capabilities are classified as state secrets and are not made available to the international community, which further increases the difficulties of this endeavour.

In conclusion, these European nations actively navigate geopolitical, technological and strategic factors to enhance cyber resilience, while the European Union supports and develop new path for the \textit{cyber revolution}. The lessons learnt from the war in Ukraine underscore the importance of flexibility, collaboration and adaptability in the face of emerging cyber threats, as they have also speeded up ongoing processes to increase cyber preparedness. The European Union's role as a normative cyber power, coordinating efforts among member states, adds another layer to this evolving landscape. The ongoing commitment to addressing challenges and fostering cross-sector collaboration will shape the future effectiveness of these nations in the dynamic cyber domain. 
