\chapter{The securitisation of Cyberspace}


Cybersecurity has grown in importance on a global scale as technological advancements have made it simpler for cybercriminals to carry out attacks across national borders. In the past, governments focused on securing their own networks and systems because cybersecurity was primarily considered as a local issue. The interconnectedness of the modern world, however, implies that cyberattacks may have worldwide repercussions. The increase in cyber threats, such as state-sponsored hacking, ransomware attacks, and intellectual property theft, has compelled nations all over the world to take steps to strengthen their cybersecurity posture. 
In this chapter, we will explore the academic literature to search for definitions as well as to define the state-of-the-art in the academic community on the interaction between cybersecurity and international politics. In the context of this study, it is necessary to provide preliminary definitions for fundamental concepts such as cyberspace, cybersecurity, and cyber policy (or cyber diplomacy and governance). However, due to the relatively recent emergence of these domains, there is a lack of widely accepted definitions. Therefore, a precise comprehension of these terms is essential to ensure a coherent and unambiguous interpretation of the entire study. 

\section{The Language of Cyber: Defining Key Terms}


\textbf{Cyberspace} The Joint Chiefs of Staff publication on Cyberspace Operations, issued by the Department of Defense (DOD), provides a definition of Cyberspace as a \textit{global domain within the information environment consisting of the interdependent networks of information technology infrastructures and resident data, including the Internet, telecommunications networks, computer systems, and embedded processors and controllers} \autocite[GL-4]{jointchiefsofstaff_2018_cyberspace}. It is notable that this definition concentrates on the physical and software-based components of cyberspace, as well as the data that resides within it and the network connections that bind them together. Information technology infrastructure, data stored and communicated through the infrastructure, and links and nodes existent in the physical domains are all necessary for cyberspace operations. Three interrelated levels make up cyberspace: the physical network, the logical network, and the cyber-persona. The IT hardware and equipment found in the physical domains that allow for the storage, transmission, and processing of data in cyberspace are included in the physical network layer. The network elements that are controlled by logic programming make up the logical network layer. The user accounts of the network or IT systems, whether they are operated by humans or machines, and their connections to one another make up the cyber-persona layer \autocite{jointchiefsofstaff_2018_cyberspace}

\vspace{0.2cm}
\textbf{Cybersecurity} The definition of Cybersecurity by the International Telecommunication Union encompasses a range of measures, including tools, policies, guidelines, risk management, and training, that are used to safeguard the assets of organizations and users in the cyber environment \autocite[2]{internationaltelecommunicationunion_2008_recommendation}. 

\vspace{0.2cm}

\textbf{Cyberpolitics} Defining cyberspace and the link between international politics and economics is a rather tricky task since it requires an analytical ability to consider numerous aspects, not only political and economic but also technical, requiring multidisciplinary skills. Academic literature has not yet found a commonly accepted definition of cyberaffairs, but it has fragmented it into different shades, linking it to suffixes that identify its meaning. For instance, according to \textcite[4]{nazlichoucri_2012_cyberpolitics}, “Cyberpolitics is a term that has emerged recently, and it describes the intersection of two distinct realms - human interactions related to the distribution of resources and power (politics), and the utilisation of virtual space (cyber) as a new arena for this distribution, with its own characteristics and dynamics”. The complexity of the cyber domain underscores the necessity for multidisciplinary research and a shared framework that can facilitate scientific inquiry in this field. 

\section{Does cyberwar exist?} 
In recent years, scholars have been criticised for their lack of attention to cybersecurity from an international relations perspective. \textcite{kello_2013_the}  argues that this gap has led to a failure to adequately explain, model, or predict cyber competition, resulting in a mismatch between theory and reality and hindering conceptual development in international relations. Given the severity of cyber events and the potential for severe reprisals outside traditional war definitions, this neglect is particularly troubling. However, \textcite{cavelty_2020_cyber} note that this situation is changing as advancements in technology, politics, and science are generating a growing body of research that applies IR or security studies theory to different aspects of cybersecurity. Therefore, it is crucial to examine this emerging literature on the subject. 


\textcite{reardon_2012_the} conducted a literature review on cyber international relations from 2001-2010, covering 49 articles related to cyberspace and information technology published in 26 major policy, scholarly IR, and political science journals. The search resulted in five issue areas: global civil society, governance, economic development, effects on authoritarian regimes, and security, with three unifying themes emerging: defining relevant subject analysis, the transformative power of cyberspace, and the mutually embedded relationship between technology and politics. These themes can provide a framework for future research in cyber international relations, which needs greater cooperation between nations at an international level to ensure effective global cybersecurity measures. Overall, the security issue remains the focus, with realist perspectives focusing on how states use cyber capabilities to achieve their objectives, and constructivists exploring national security as an intersubjective interpretation.
The securitisation aspect is a crucial viewpoint for what regards cybersecurity in IR. Mainly, the discussion between academia and the military has had as its subject the concept of cyberwar. 

According to certain academics, the term cyberwar is misleading because it suggests a level of violence and devastation that is not always present in cyberattacks. For instance, \textcite{rid_2017_cyber} believes that rather than outright destruction, cyber threats frequently concentrate more on espionage, theft, or disruption. The extensive use of these attack vectors constitutes a non-violent confrontation, which will replace the violent realm of the international arena \autocite{rid_2017_cyber}. This thesis is supported by identifying the most sophisticated cyber-attack to date, Stuxnet, as an act of cyber-enabled stand-alone sabotage is not connected to a conventional military operation. As \textcite{maschmeyer_2022_goodbye} outlined in the Ukraine Case Study, cyber operations are either too slow, too weak, or too volatile to provide significant strategic value in hybrid conflict and war. Nevertheless, it should be noted that the potential for cyber operations to have tangible, physical disruptions cannot be dismissed, as happened with the Stuxnet attack.

The international community has been divided when it comes to determining whether a cyberattack can be classified as a violent act, carrying the same legal significance as an armed attack. Indeed, it becomes important to define the concept of war beyond its often-sensationalised portrayal in the media. By identifying the specific characteristics that underpin this phenomenon, it could be useful to assess possible links with the cyber domain.

Nowadays, the Atlantic Alliance and its partners have developed the Tallinn 2.0 Manual, as the first tangible achievement of International Law applied to cyberspace. In this document, a cyber operation could be an act that implies the use of force only in determinate scenarios. Firstly, \textit{it is not the instrument used that determines whether the use of force threshold has been crossed, but rather […] the consequences of the operation and its surrounding circumstances} \autocite[328]{a2017_tallinn}. In addition to the effect of the act itself, the target plays a central role in determining the consequences and the threshold. Specifically, an attack on critical infrastructure has a more disruptive effect compared to other attacks.  However, the manual does not include the perspective of those states that are outside the NATO vision and specifically highlights the difficulty among international experts in the application of international law to cyberspace. In fact, the main issue concerning the adaptation of international law to cybersecurity within international organizations lies in the opposition of China and Russia towards adopting substantial resolutions. In this regard, the European Union could be recognized as a reliable example of how the different approach of its member states on the application of international law in cyberspace does not preclude the possibility to cooperate efficiently in this field \autocite{delerue_2023_toward}. One-third of the EU members have proposed a path towards it. Even with differences, it is recognisable a significant degree of convergence among the approaches. 

Waiting for further developments in the international scene, academics have tried to outline how and if cyberattacks can be associated with an armed attack.

Historically, Carl von Clausewitz’s ideas about the characteristics of war have influenced military schools and universities globally. Clausewitz viewed war as a multifaceted and complex phenomenon that involves military, political, and psychological factors. He famously described war as \textit{the continuation of politics by other means,} highlighting the close connection between warfare and politics  \autocite{carlvonclausewitz_1984_on}. Due to this view, his perspective of war emphasised its intricate nature, its interdependence with politics, and its violent and unpredictable characteristics. Clausewitz’s ideas are also central to the cyber realm. Firstly, it is important to consider that cyber operation is an atypical way to pursue political objectives, but less expensive and recognisable as war. Therefore, scholars should assess a central question in regard to the violent act that is so often associated with traditional forms of conflict, as their applicability to the cyber domain is not evident yet.

Furthermore, the relationship between the cyber domain with the other classical domain contributes to explaining the effects of cyber operations with the aim to produce violence. As \textcite{worldeconomicforum_2023_the} reveals in the\textit{ Global Risk Report 2023}, cyber is strongly related to the increased risk of polycrises, where multiple crises interact and have a more significant impact than the sum of each part. Hence, the cyber domain could be an important domino piece that could trigger unexpected crises in other domains. 

However, other scholars contend that the idea of cyberwar provides a legitimate and practical framework for comprehending and confronting cyber dangers. They asserted that cyberattacks may have serious repercussions and may be as harmful as conventional military assaults. According to \textcite{kello_2013_the}, technological advancements have brought about a significant impact on the political framework of international society. The effects can be seen in the way new technologies have empowered previously unrecognised players with subversive motives and aims. As with the advent of the atomic bomb, the cyber revolution provoked the transformation of the Hobbesian Framework, which focuses on the state survival principle. Differently, the cyber revolution offers the possibility for non-state actors to inflict economic, social, and physical damage. Therefore, cyber operations have been applied on the battlefield, as happened in Ukraine in 2015 and 2022. Russian-backed hackers have taken off the power grids, banking systems, and electoral platforms in order to disrupt civil society welfare. The grey zone between war and peacetime is one of the main issues regarding cyber operations, since it does not allow the application of basic international law and adds more complexity to this problem. For this reason, influence operations and cyber ones, simultaneously with low-intensity military activities, have real effects on society, but without making possible an effective short-term tactical defence for the victim. 

Nevertheless, the Russian invasion of Ukraine has unfolded how cyber capabilities could be applied in wartime. The cyber dimension of the war in Ukraine has garnered significant attention and interest among researchers and analysts. Numerous studies have delved into the intricate patterns and evolving trends of cyber-attacks, assessing the strategic effectiveness of cyber operations, and contemplating broader implications for future conflicts. However, in this conflict, destructive cyber operations were not exploited to cause considerable damage or at least they were defeated by the defensive posture of Ukraine and its allies. It is crucial to ensure that the cyber operations that occurred during this conflict are the first one of the first cases of cyber capabilities used during wartime between two state actors. Even if the ineffectiveness of the attacks can be attributed more to the military and political structural problems of the Russian Federation rather than solely to the cyber operations themselves, they serve as a starting point to think more about what directions the cyber national policies are going and what this conflict represents for the future.

Hence, the cyber operations effects incur mostly on the civilian side, in private companies, and during peacetime. These characteristics do not allow us to determine if and when a cyberwar takes place. Differently, scholars and practitioners are using mostly the term cyber operations as the type of or a part of a military operation in which cyber weapons/capabilities are used to achieve military objectives in front of adversaries inside and/or outside cyberspace \autocite[35]{maathuis_2018_developing}. For this reason, the war suffix should be avoided, to create less ambiguity for decision-makers and future international regulation on cyberspace and its activities. 

Just as it occurred during the Cold War, technological advancements in the military realm have triggered a race for innovation in this domain, aiming to attain the leverage of deterrence necessary for the survival of the state. Given the constantly evolving nature of cyber threats, deterrence measures against cyber operations are even more difficult to build up. The comparison of cyber deterrence to nuclear deterrence is not valid because every state or non-state actor theoretically could launch a cyber operation, but not a nuclear attack. Therefore, the distinct characteristics of each type of attack preclude a direct comparison between the two.  Deterrence, understood as the ability \textit{to prevent from action by fear of consequences} (Schelling, 1966 cited in \textcite{nye_2017_deterrence}), seems challenging to apply in cyberspace. As \textcite[54]{nye_2017_deterrence} observed, there are four types of deterrence and dissuasion methods that could be applied in cyberspace, considering all the limitations and the problems (i.e., attribution) that this domain could bring. 

\vspace{1cm}
\begin{table}[htbp]
  \centering
  \caption{\emph{Overview of Major Deterrence Methods in Cyberspace based on Nye (2017)}}
  \begin{tabular}{p{3cm}p{9cm}}
    \toprule
    \textbf{Deterrence} & \textbf{Description} \\
    \midrule
    Punishment & Imposing sanctions or other consequences on a state that has committed an attack, such as economic penalties, diplomatic isolation, and military action. \\
    Denial & Hardening one's own systems and networks to make them more difficult to penetrate, and thereby denying an adversary the ability to achieve their objectives. \\
    Entanglement & When countries' systems are interconnected, a cyberattack on one country can affect other countries as well. This connection can discourage countries from launching cyberattacks because they might harm their own systems in return. \\
    Norms & Establishing international norms and rules of behaviour for cyberspace and making it clear that certain types of activity are unacceptable and will not be tolerated. \\
    \bottomrule
  \end{tabular}
\end{table}
\vspace{1cm}

Retaliation and punishment deterrents have been less applicable in cyberspace as a deterrence method for the attribution problem of cyber operations. For this, states’ strategies have replicated a more defensive approach, trying to increase the barriers for the attackers and make cyber operations more costly and time-consuming. However, the complexity and fragmentation of the defence and the authority make the defence strategy not applicable to all possible victims, but only to those infrastructures that are considered strategic. Given the interdependence of economic, political, and military sectors, new strategies are required to achieve preventative protection against future cyber operations. One such approach involves the application of international law to the cyber domain. 


The development of cyber norms, concepts, and regulations as well as the difficulties in putting them into practice and upholding them have all been studied in cybersecurity research. \textcite{simi_2021_cyber} legal approach to the study of international relations through agreements made between states in cyberspace over the last 20 years is one such study. The research investigates how the information revolution has affected international relations, the difficulties and successes in cybersecurity and cyber activity legal control, and how much traditional diplomacy has changed in modern cyberspace. According to Simić, state national interests will continue to dominate cyberspace's legal status as a grey zone. This leads to the notion that the theory of realism dominates over the liberal theory of international relations in this new area, which implies cooperation on a global level through international organizations \autocite{simi_2021_cyber}.


Nevertheless, securitisation is not the only lens through which cyber affairs can be analysed. The cyber revolution has critical implications for the economy and societies, making it difficult for policymakers to keep pace with the rapid growth of technology.  


