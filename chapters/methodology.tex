\chapter{Methodology}

This chapter examines the methodology used in the study, focusing on the impact of cyber operations in Ukraine on the EU cybersecurity framework. The methodological chapter is structured to provide an understanding of the complex relationship between cyber operations, warfare, and cybersecurity policy at national and EU levels The chapter will cover the research design, compilation and description of datasets to be used. In addition, it will describe the methods used to analyse data, its integrity and reliability, and the conduct of a qualitative document review.

\section{Research Desing}

The goal of the study is to investigate the cyber aspect of the conflict in Ukraine and collect important lessons for the cyber strategy of the European Union. In order to accomplish this goal, a study of the cyberwar in Ukraine from both the Ukrainian and Russian perspectives will be conducted, revealing the strategies employed to deter cyberattacks during wartime. Then, these methods will be researched on Estonian, French, German, and Dutch national cybersecurity policies. We will additionally investigate how the European Union may act as a catalyst for better cybersecurity among its Member States by encouraging and supporting the adoption of these strategies.

This research will employ a mixed-methods approach. The decision to proceed with both qualitative and quantitative analysis is informed by the complex and multifaced nature of our research question and domain. The quantitative data will provide us with statistical rigour with the aim of identifying patterns and correlations between observations. On the other hand, the qualitative analysis will help us to contextualise and interpret observed data. Qualitative methods, such as normative content analysis and case studies, will be used in our analysis. 

The research will be divided into two parts. The first will focus on the cyber dimension of the war in Ukraine, where it will serve as a case study to analyse the adoption of cyber operations during wartime. It will investigate the methods, objectives, and effects of cyber operations.  

The second part will delve into the analysis of the European Union and its Member States. Its position will delight its status as a cyber power and the current efforts to build a cyber defence for its Member States. The different methods of cyber defence within the European Union will also be addressed by comparing France, Germany, the Netherlands, and Estonia. The comparative analysis will take into consideration factors such as the national institutions on cybersecurity issues, EU-sponsored projects under the Permanent Structured Cooperation (PESCO) and insights from previous cyberattacks.

The reason for the selection of these countries lies in their strategic role in cyber defence within the European Union, even with different approaches. The strategic differences represent a spectrum between experienced cyber defenders (Estonia) and major players in EU politics (Netherlands, France, and Germany). Additionally, these nations present a rich repository of official documents and publicly accessible data, significantly contributing to our research.

\section{Research Type}

Our work falls within the category of interpretive research. We aim to not only describe and understand the key lessons learned, but also explain why these lessons are essential for shaping future strategies. The interpretive research is well-suited for emerging fields such as cyber politics during wartime. A flexible and open-ended research method as well as a multidisciplinary approach are required due to the complexity of the area. The interpretative research will not find the causal phenomena for generalizable law-like regularities \autocite{lamont_2021_research}, instead, the focus will be on understanding the interaction between politics and technology in a fast-paced world.

\section{Data Collection}

When studying cybersecurity issues, one of the biggest challenges that scholars and researchers must tackle is associated with data availability. Different factors affect the availability of data on cyberattacks, perpetrators, and victims. \textcite{cremer_2022_cyber} analysed 5219 cyber peer-reviewed studies and found that only 79 unique datasets were used. First, cybersecurity, like other security-related areas, is hidden behind states' or companies’ secrets. The state is not always open to disclosing attacks on its critical infrastructure, which undermines its popularity among citizens. However, cyberattacks undermine their reputation. Even in most countries, there are some binding rules to inform companies’ clients about data breaches or other types of attacks; information about attacks is fragmented and incomplete. Another reason relates to the nature of cyber-capabilities. Before being discovered or starting to produce some effects, a malicious attack vector can hide for an indefinite time. For instance, Stuxnet, a highly sophisticated cyberattack, was concealed through a combination of intricate techniques and targeted strategies to attack Iran's uranium-enrichment centrifuges. The gradual and methodical approach that Stuxnet took in damaging the centrifuges allowed it to operate silently within Iran's nuclear facilities for several years before it was uncovered. Its inadvertent spread beyond its intended target in 2010 led to this discovery. 
Furthermore, the number of cyberattacks is increasing exponentially, posing a challenge for researchers who may struggle to trace and analyse all of them. 

Moreover, there is a lack of open-source databases containing information on cyberattacks \autocite{cremer_2022_cyber}. Many companies specialising in cyber threat intelligence, including Recorded Future, CrowdStrike, and Fortinet, among others, often refrain from publicly sharing such data because of their business models. However, open-source databases could help practitioners and companies facilitate data gathering and sharing to develop and upgrade their best practices to mitigate cyberattacks. In addition, an open-source database of cyberattacks could contribute significantly to increasing societal awareness of cyber issues. 

Some steps in this direction have been taken by \textit{open-source intelligence} (OSINT) communities, which actively work to collect and share cyber incident information. These efforts play a crucial role in increasing transparency and knowledge in cybersecurity. 
Two datasets will be used for this purpose of the research project: the \textit{Cyber Operations Tracker} created by the Council on Foreign Relations and the EU Database from the \textit{European Repository of Cyber Incidents} (EuRepoC). 

The use of two different datasets for statistical analysis is essential for several reasons. First, it helps validate and cross-verify findings, reducing the risk of drawing incorrect conclusions from potentially biased or incomplete data. Second, it allows for a more comprehensive analysis by providing diverse perspectives and insights, thus enabling a deeper understanding of the subject. In addition, it facilitates comparative analysis, enabling researchers to explore variations and trends across different datasets, which can lead to more robust and reliable research outcomes.

\newpage
\textbf{Cyber Operations Tracker}

The \textit{Cyber Operations Tracker} developed by the Council on Foreign Relations, an independent think tank, tracks state-sponsored cyber operations which have been disclosed publicly from 2005 to nowadays. The dataset does not include hacktivists' actions and other threat actors that are not engaged in denial-of-service attacks, espionage defacement, destruction of data, sabotage and doxing \autocite{councilonforeignrelations_2023_tracking}. The sources of the attacks are available openly and mentioned in the dataset itself. The Council on Foreign Relations updates the dataset quarterly. As mentioned in their methodology, there are some limitations for the dataset \autocite{councilonforeignrelations_2023_tracking}:

\begin{itemize}
    \item \textit{Attribution}: due to the nature of cyber operations and deceptive tactics from threat actors, attributing the attack to an entity is a challenge. Even though, the dataset includes multiple sources, from technical sources to government websites. However, it remains a contentious issue.
    \item \textit{Data Completeness}: as argued in the previous section, no datasets could be comprehensive to track every cyber (operations) attack. The perpetual change of the techniques and strategies of cyber operations makes the research more difficult.
\end{itemize}

\textbf{EuRepoC}

The \textit{European Repository of Cyber Incidents} (EuRepoC) is an independent research consortium that provides a well-structured overview of cyber incidents with detailed information from 2001 to nowadays. The consortium was established in 2022 by researchers at the University of Heidelberg, the Cyber Policy Institute, the University of Innsbruck and the German Institute for International and Security Affairs (SWP). The EuRepoC dataset combines three coding processes: technical coding by IT forensic experts evaluating various aspects of cyber incidents, political coding by experts focusing on political attribution and responses, and legal coding by legal experts assessing the legal impact and justification of response options, including evidence for sanctions. 

The EuRepoC dataset's data selection criteria involve focusing on publicly reported cyber incidents that have affected political or state actors, have been associated with state actors, or have been publicly politicized. This approach deliberately excludes cases that concern specific stakeholders but are not prominently addressed by political actors. Additionally, the dataset includes a cyber intensity scale that assesses the coded incident types, their potential physical effects, and socio-political severity, assigning a weighted score ranging from 1 to 15 to provide a comprehensive evaluation of each incident's impact and significance.

Both datasets are similar in their approach, and most of the data overlaps. However, the choice to analyse both is useful to understand both a global (with Cyber Operations Tracker) and European (with EuRepoC) perspective of cyberattacks. In addition, the EuRepoC gives us information on whether the attack has caused disruption or not, which is useful to compare the ability of countries to counter the attacks. 

Furthermore, for insights on cyber operations in Ukraine, data from non-governmental organizations (NGOs) such as the CyberPeace Institute and technology companies such as Microsoft and Google (Mandiant). The reason for using data from such sources is linked to their involvement in the cyber defence in Ukraine. Specifically, CyberPeace Institute has developed a project which discovers cyberattacks that aim at civilians while assessing their impact. In addition, Microsoft and Mandiant, among other companies, have contributed to the technical cyber defence of Ukraine against Russian cyber operations \parencite{huntley_2023_fog, smith_2022_defending}.

Moreover, other data from cyber indexes will be used to assess the cyber power of selected EU countries. 




To assess the cyber power and capabilities of the selected European Union countries (France, Germany, the Netherlands, and Estonia) and to facilitate comparative analysis, you have chosen several cyber-related indexes:


\textbf{National Cyber Power Index}

The \textit{National Cyber Power Index} developed by the Belfer Center for Science and International Affairs at Harvard Kennedy School was published in 2020 and updated in 2022. The index evaluates various objectives, including surveillance of domestic groups, enhancing national cyber defences, controlling the information environment, foreign intelligence collection, advancing technology competence, disrupting adversary infrastructure, setting international cyber norms, and accumulating wealth or cryptocurrency. These objectives provide a holistic assessment of a state's cyber capabilities, enabling scholars to analyse them in conjunction with other national tools, such as military and diplomatic means, for pursuing national interests. 

\textbf{Cyber Arms Watch}

The Cyber Arms Watch's \textit{Cyber Transparency Index,} developed by The Hague Centre for Strategic Studies, and published in 2022, evaluates the transparency of a state's cyber offensive capabilities using two main components: the Declared Capabilities Rating (DCR) and the Perceived Capabilities Rating (PCR). The DCR assesses how much a state publicly discloses about its offensive cyber capabilities, while the PCR gauges the perceived capabilities based on external sources. These ratings create the Cyber Transparency Index, which categorizes 60 countries' cyber transparency, including European nations and major international actors. Differently from the other index that included diplomatic and economic cyber power instruments, the Cyber Arms Watch focuses exclusively on offensive cyber capabilities. 


\textbf{National Cyber Security Index}

The \textit{National Cyber Security Index} (NCSI) measures the preparedness of countries to counter and manage cyber threats. It aligns with national cybersecurity frameworks and focuses on measurable aspects of cybersecurity implemented by central governments. The NCSI employs structured categories, capacities, and indicators, all assigned values reflecting their importance. These categories are the general cybersecurity indicators, such as policy development and threat analysis; the baseline cybersecurity indicators, such as the protection of digital and essential services; and incident and crisis management. Country ratings are based on publicly available evidence like legal acts and official documents. The NCSI Score represents a country's performance relative to the maximum, scaled to 100\%. It also includes the Digital Development Level (DDL) to show the relationship between cybersecurity and digital development. In addition, data is collected through government contributions, organizations, individuals, and the NCSI team, with rigorous expert reviews. The NCSI is continually updated for accuracy and relevance.

\begin{table}[htbp]
  \centering
  \caption{Comparison of Cybersecurity Indexes}
  \begin{tabular}{p{1.2cm}p{2.5cm}p{2.5cm}p{2.2cm}p{2.2cm}p{3cm}}
    \toprule
    \textbf{Index Name} &
    \textbf{Objective} &
    \textbf{Data Source} &
    \textbf{Publication Year/Update Frequency} &
    \textbf{Limitations} &
    \textbf{Relevance to Research} \\
    \midrule
    \textit{National Cyber Power Index} &
    Assess cyber capabilities &
    In-house data, proxies &
    2022/Updated regularly &
    Difficulty in data collection for certain objectives &
    Evaluate cyber capabilities \\
    \midrule
    \textit{Cyber Arms Watch} &
    Analyze transparency of cyber offensive capabilities &
    The Hague Centre for Strategic Studies &
    2022 &
    Challenges in assessing transparency &
    Measure transparency and disclosure \\
    \midrule
    \textit{Nation Cyber Security Index}&
    Measure preparedness to prevent cyber threats &
    Publicly available evidence materials from governments, organizations, and individuals &
    2016/ (Continuous process) &
    Data collection and public evidence-based assessment &
    Assess national cyber security preparedness \\
    \bottomrule
  \end{tabular}
\end{table}


The relevance of the indexes will not be on the rankings or the numerical value for each country. Rather, the picture that those indexes represent. Those indexes, even with different approaches, are powerful tools in situations in which retrieving information from sectors where its security is a must. 

\section{Data Analysis Methods}
\subsection{Data processing}

Data pre-processing will be carried out using Python in Jupyter Notebooks, and all \textit{.csv} and \textit{.ipynb} files will be made freely available on \textit{GitHub}, where the documentation for each Python script will be present. We will utilize Python libraries such as Pandas, Matplotlib, Seaborn, Folium, Plotly, NetworkX, and natural language processing tools to bolster our data analysis and visualization efforts.

To narrow our focus to the selected countries, we will filter both cyberattack datasets. For the Cyber Operations Tracker database, where country references are absent and only entities or institutions are mentioned, we will leverage natural language processing techniques to isolate the designated countries. Additionally, columns with null data will be removed. A consolidated dataset will be created to incorporate the indexes mentioned earlier, and it will be merged with statistics gathered from national authorities, ENISA, and European institutions (such as PESCO projects, investments in ICT and others). We aim to investigate the methods, objectives, and effects of cyber operations in this context, shedding light on the evolving landscape of cyber warfare in the region.

For the effective visualisation and communication of our research findings, we will employ various scripts to create maps, network diagrams, and statistical plots. Statistical methods, such as regressions and correlations, will be carried out with the aim of finding relevant patterns and insights on our exploratory data analysis.


\subsection{Data Validity and Reliability}

To ensure the quality of data, the research presents only reliable and trusted sources, avoiding speculations. The data pre-processing will enhance the validity of our datasets while handling missing values. In addition, the interpretation of the data will be supported by the literature review. 

\subsection{Qualitative Document Analysis}

The content analysis will be conducted on European official documents obtained through EUR-Lex. The documents will be classified according to their nature (regulations, directives, information among others), and analysed with the aim to explore EU objectives and priorities, policy measures and reference to the cyber capabilities. 
The list of documents to analyse is present in the Annex. The analysis will start from 2014, the year of the first Cybersecurity Strategy of the European Union, to 2023. For achieving a systemic analysis, we will use a common framework for each of the documents presented in Table \ref{tab:framework-elements}

\begin{table}[ht]
\centering
\caption{Framework Elements for Analyzing EU Cybersecurity Documents}
\label{tab:framework-elements}
\begin{tabular}{|c|p{8cm}|}
\hline
\textbf{Framework Element} & \textbf{Description} \\
\hline
\multirow{2}{*}{\textit{Document Classification}} & Categorize documents based on their nature (e.g., regulations, directives, reports, strategies). \\
& \\
\hline
\multirow{2}{*}{\textit{Objectives and Priorities}} & Extract stated objectives and priorities outlined in the documents. Consider overarching objectives and specific cybersecurity goals. \\
& \\
\hline\textit{}
\multirow{2}{*}{\textit{Cyber Capabilities}} & Examine references to the EU's cyber capabilities. Look for mentions of cybersecurity capabilities in terms of defense, response, and deterrence. \\
& \\
\hline
\multirow{2}{*}{\textit{Roles and Responsibilities}} & Identify roles and responsibilities assigned to EU bodies, member states, and stakeholders in implementing cybersecurity measures. \\
& \\
\hline
\multirow{2}{*}{\textit{Resource Allocation}} & Investigate how the documents address resource allocation for cybersecurity initiatives. Consider budgetary provisions, funding mechanisms, and priorities. \\
& \\
\hline
\multirow{2}{*}{\textit{Challenges and Threats}} & Identify challenges and emerging threats mentioned in the documents. Consider the EU's plans to address these challenges and adapt to new threats. \\
& \\
\hline
\end{tabular}
\end{table}

\section{Limitations}

What do information on cyberattacks, the use of cyber tactics, and how indexes are made have in common? The frequent difficulty in obtaining thorough, current information. Cyberspace, which is human-made, provides the perfect setting for keeping a lot of information secret. In fact, the secrecy surrounding such actions frequently makes it difficult for academics to locate trustworthy sources and data. 

The countries that were chosen are those that promote greater transparency in government files. This method, however, presents a geographical bias because these states cannot be seen as unique role models for the European Union but rather as the most cyber-developed countries in this area.

However, it is crucial to delight some of the problems for research in the cyber domain, particularly concerning the application of cyber operations during wartime with the Ukraine War case. The limitations of this work are reflected in the availability of data. Both parties involved in the conflict were restrained from sharing them. In addition, the \textit{Cyber War Fog} could affect the objectivity of critical analysis. The Ukraine–Russia war could lead to imperfect empirical knowledge of a single historic event.  Despite all the limitations, this research could serve as a springboard for further investigation into the Ukrainian conflict's cyber dimension, which will serve as a foundation for other studies on cyber operations in times of war.

\section{Ethical Considerations}

All information and data gathered for this research are freely available on the internet. Due to sensitive information that cyber operations involve, the research will not use potentially classified data, but rather will use only open-source data and peer-reviewed information. 




