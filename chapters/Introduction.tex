\chapter*{Introduction}
\phantomsection
\addcontentsline{toc}{chapter}{Introduction}


\begin{quote}
\centering
--------------------------------------------------------------------------------- \\
If you spend more on coffee than on IT security, you will be hacked.
\\ - Richard A. Clarke
--------------------------------------------------------------------------------- \\
\end{quote}

Today's world is characterized by rapid technological advancement and interconnected global networks, wherein the security landscape is constantly evolving. The cyber domain, as a cross-cutting sphere, serves as a meeting ground for international actors, private companies, and non-state entities, where collaborations and occasional struggles unfold. The absence of a deterministic international law applied to govern and prevent escalation leads to a chaotic and unexpected outcome in the real world.

In recent times, scholars' attention to cybersecurity issues has significantly increased. Cyber issues are no longer marginalised in academic discussions and are not solely viewed through the lens of security from an Anglo-Saxon perspective. However, the academic community has yet to reach a consensus on the fundamental concepts of the matter; theories, frameworks, and analyses have been produced with significant results, laying the groundwork for the future growth of this domain.

This research delves into the dynamics of cyber defence by analysing how cyber operations are conducted, focusing on the Russia-Ukraine War, European Union and its Member States. 

This research project will exploit the characteristic and the outcomes of the Russian invasion of Ukraine, analysing it from a cybersecurity perspective to takeaway insights and lessons from the first interstate war in the information age. 

The research will encompass a multidisciplinary approach drawing on insights from International Relations, Cybersecurity, European Law and data analysis through programming languages (\textit{python}). Specifically, the research gathered data from dataset such as the European Repository of Cyber Incidents (EuRepoC) and Cyber Operations Tracker from Council on Foreign Relations. 

The theoretical framework of the research draws on concepts such as Complex Interdependence, Securitisation Theory, and Neo-Functionalism, providing a lens through which to analyse the multifaceted nature of cyber-related phenomena. Each theoretical perspective contributes unique insights to the understanding of cyber power dynamics, the securitisation process, and the evolving nature of state interactions in the digital age.

The subsequent chapters navigate through critical aspects of cyberspace, exploring key terms, assessing cyber power, and examining the significance of cyber defence. The research further investigates the cyber dimension of conflicts, specifically focusing on the Russian invasion of Ukraine, where the cyber realm constitute a real threat. It sheds light on intelligence gathering, the role of hacktivists, and the lessons to be learned from the cyber operations during wartime.

The latter chapters transition towards a European perspective, evaluating the EU's historical cyber strategy, cooperation mechanisms between institutions and member states, and the overarching theme of digital sovereignty. The research culminates in a comparative analysis of European countries—France, Estonia, Germany, and the Netherlands—assessing their preparedness for cyber operations.

In conclusion, the comparative analysis underscores diverse approaches to cybersecurity, shaped by the lessons learned from the Ukraine conflict. Economic measures, critical infrastructure protection, and military integration are identified as crucial pillars for cybersecurity resilience across these nations. Challenges such as the skills gap and public-private sector cooperation persist, prompting tailored initiatives and collaborations. The lessons drawn from the war in Ukraine emphasise the necessity for flexibility, collaboration, and adaptability in the face of emerging cyber threats. The commitment of these European nations, guided by ongoing processes to increase cyber preparedness, signifies their active engagement in shaping the future effectiveness of their cybersecurity efforts. 