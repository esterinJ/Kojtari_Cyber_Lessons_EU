\chapter{Theoretical Framework}

Our theoretical foundation to pursue this research will be based on three different theories in International Relations. The complex interdependence and subsequently weaponization of the interdependence will explore the role of cyberspace in the hyper-globalised world and the growing relevance of non-state actors. Next, we will apply securitization theory to cyberspace, now recognized as the fifth battlefield, where zero-sum games are played. Finally, the institutionalism theory will explain why the European Union is relevant in shaping the member states cyberstrategies and how it is contributing to set new rules for the other actors. 

\section{Complex Interdependence and its weaponization}

The theory of Complex Interdependence was minted by Joseph Nye and Robert Keohane in the 1970s with their book Power and Interdependence. They understood that transnational political issues were linking states. Specifically, they focused on the flows of transactions, goods, and services to create boundaries in the international arena. As \textcite{nye_1971_transnational} asserted, the theory of Complex Interdependence is characterised by:

\begin{enumerate}
    \item \textbf{Multiple channels between actors}: These channels encompass diplomatic, economic, social, and technological interactions.
    \item \textbf{Absence of hierarchy}: new actors move in the international arena, such as the Non-Governmental Organisations. 
    \item \textbf{Minor role of Military force}: while military power remains important, it is not the only one for pursuing the state’s interests. 

\end{enumerate}

In the information age and an ever-more globalised world, the channels through which Complex Interdependence operates have significantly expanded and evolved. These transformations have profoundly reshaped the dynamics of international relations. Interdependence can be weaponised by one of the actors. In the context of cyberspace today, even small states and non-state actors can render larger states vulnerable \parencite{farrell_2019_weaponized, nye_1971_transnational}(Farell and Newmann, Nye and Keohane). Nowadays, cyberspace constitutes the basis of any economy in the world, from international trade to financial transactions and public information. All data is no longer physically stored in folders but in the cloud database, thus reducing transaction costs. Although theoretically, each actor could take advantage of these technological innovations, these benefits are distributed non-homogeneous among them. The US, Russia, China, and the European Union are the major international powers from the technological infrastructures' perspective, as well as the ability to create leading international companies in the tech sector. Digital Information and Communication Technology (ICT) and technological innovations are the main reasons for the rapid spread of global supply chains \autocite{mansfield_2021_embedded}, fortifying interdependence and enhancing productivity. Different poles of cyber power but interdependent with each other. Interdependence is structured from the material perspective (China is the leading exporter of rare earth elements, indispensable for constructing semiconductors) and technological know-how. Indeed, this work will go beyond the liberal complex interdependence by affirming that network interdependence provokes power imbalances among states; consequently, asymmetric network structures create the potential for weaponised interdependence \autocite[45]{farrell_2019_weaponized}. Cyberspace and its physical infrastructures are asymmetric distributed and decentralised networks, but state actors are using its interdependence to project their power. However, cyberspace weaponisation is possible only for privileged actors who control hubs (physical such as satellite and submarine cables and online as servers) and have domestic institutions with the cyber capabilities to intervene \autocite{farrell_2019_weaponized}.

\section{Securitisation Theory}

The concept of securitisation was developed by the Copenhagen School as one of the milestone of security studies discipline. In 1998, Barry Buzan, Ole Wæver, and Jaap de Wilde published "Security: A New Framework for Analysis," which not only defined the concept of security but also continues to shape it to this day. This concept of securitisation has an intersubjective meaning, where a state can socially construct a threat, even if it is not necessarily reflected by the objective, material circumstances of the world \autocite{balzacq_2016_securitization}. This approach has been applied to other areas such as the economy, energy, and environmental issues, where each sector poses a threat to a referent object, such as identity or ecosystem \autocite{eroukhmanoff_2018_securitisation}. The fundamental aspect of the theory lies in the capacity of political discourse, a speech act, to not just describe reality but to constitute it. 

In the digital era, cyber constitutes another field where securitisation has been applied.  Scholars such \textcite{hansen_2009_digital} have assumed that this new field has experienced \textit{hypersecuritisation} since the dystopic futures described in the pop culture about cyber doom day. The literature highlights the new capacity of non-state actors to have similar \textit{cyber-fires} of states and moreover, the possibility of individuals to have a say, without underestimating the network that englobes all the actors. 

With this regard, \textcite{lacy_2018_securitization} have defined three different positions that shape the securitisation of cybersecurity.

\begin{itemize}
    \item \textbf{Cyber Catastrophist}: This perspective asserts that there's a genuine risk of digital catastrophes that could cause major societal, economic, and infrastructure problems. It contends that discussions about geopolitical threats often downplay the significance of cyber threats and caution against potential events akin to \textit{cyber 911s} or \textit{cyber–Pearl Harbors}. However, it doesn't foresee cyber disasters causing the same level of violence as weapons of mass destruction.
    \item \textbf{Digital Realist}: This viewpoint highlights the significance of concentrating on both offensive and defensive cybersecurity, as well as strategies for cyber warfare. It advises against viewing cyberweapons as revolutionary tools that would reshape the future of warfare and bring about catastrophic scenarios resembling those in science fiction movies. The argument is that there are always technical solutions available to protect critical infrastructure, and that non-state actors and even states lack the motivation or capability to cause cyber disasters resulting in physical harm.
    \item \textbf{Techno-Optimist}: This perspective offers a more positive outlook, suggesting that cybersecurity is an ongoing process of improvement rather than a fixed destination. It acknowledges that progress is happening, albeit slowly. Recognising the unpredictable nature of geopolitics and technology underscores the importance of continually reevaluating fundamental questions and assumptions as the landscape evolves.
\end{itemize}

In cyberspace, two formerly different notions of security merge, as technical and national become one \autocite{balzacq_2016_securitization}. Another critical point regards the transversality of cyberspace to other fields that have experienced securitisation, such as the critical infrastructure, the environment or the health sector.  These three different positions would be useful for further analysis in this research, where we will discuss cyber operations during wartime, which seems to be far from a game-changer in international relations. 

\section{Neo-Functionalism}

Neo-functionalism is a concept in International Relations theory that emerged post-World War II. It offers an approach to regional integration and transnational cooperation, focusing on the role of supranational institutions and non-state actors. This principle was influenced by the European Coal and Steel Movement within the European Union \autocite{dunn_2013_neofunctionalism}.

The theory is based on the concept of \textit{spillover}, which means that mergers in one sector can lead to synergies in related fields. For example, the creation of political mechanisms to regulate these economic relations may result from economic integration, such as the creation of trade blocs' spillover effect and the process leading to this when economic integration leads to political integration. In the decision-making and policymaking process at the member state level, the participation of international bodies and institutions such as the European Commission of the European Union greatly contributes to this, making development beyond national jurisdiction possible for integration. Neo-functionalism also emphasises functional differentiation, suggesting that integration often begins in specific areas before spreading to other areas. It recognises that different parts of the integration can develop at different speeds. For example, local integration efforts begin with economic cooperation and gradually expand to include security or environmental policy.

The theory also acknowledges the possibility of spillover, where difficulties in one area of integration negatively impact previously integrated areas. For example, political tensions among member states may impede previously achieved economic cooperation. Neo-functionalism asserts that integration processes are often driven by elites, such as political leaders and bureaucrats, who have a vested interest in promoting integration projects. These elites play a critical role in overcoming domestic opposition to integration and guiding the agenda towards deeper cooperation. Additionally, neo-functionalism recognizes functional overlap, where issues in one policy area intersect with those in another. This creates opportunities for integration as policymakers recognize the potential benefits of cross-policy cooperation.

This theoretical framework is also applicable to the European strategy on cybersecurity. In fact, the European Union has taken a strong stance by integrating cyber policies across the Union through initiatives such as the Cybersecurity Strategy, Network and Information Directive, and the Cybersecurity Act \autocite{kasper_2020_towards}. Like other areas of EU policies, the degree of integration varies due to member states' preferences and their bargaining power. However, cyberspace is now considered a common good and, therefore, a priority on the EU agenda. This prioritisation involves the private sector, which is no longer seen as a passive player, but as a proactive one that determines the rules of the game.

\section{Conclusion}

In conclusion, the theoretical frameworks presented in this chapter provide a solid foundation for our research project in the field of international relations and cybersecurity. The frameworks complement each other and will allow us to explore the interplay of complex interdependence, securitisation, and regional integration in the realm of cyberspace.